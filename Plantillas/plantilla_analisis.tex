%% LyX 2.3.6.1 created this file.  For more info, see http://www.lyx.org/.
%% Do not edit unless you really know what you are doing.
\documentclass[spanish,english]{article}
\usepackage[T1]{fontenc}
\usepackage[latin9]{inputenc}
\usepackage{float}
\usepackage{amsmath}
\usepackage{amsthm}
\PassOptionsToPackage{normalem}{ulem}
\usepackage{ulem}

\makeatletter

%%%%%%%%%%%%%%%%%%%%%%%%%%%%%% LyX specific LaTeX commands.
\newcommand{\noun}[1]{\textsc{#1}}
\floatstyle{ruled}
\newfloat{algorithm}{tbp}{loa}
\providecommand{\algorithmname}{Algorithm}
\floatname{algorithm}{\protect\algorithmname}

%%%%%%%%%%%%%%%%%%%%%%%%%%%%%% Textclass specific LaTeX commands.
\numberwithin{equation}{section}
\numberwithin{figure}{section}
\theoremstyle{definition}
\newtheorem*{defn*}{\protect\definitionname}

%%%%%%%%%%%%%%%%%%%%%%%%%%%%%% User specified LaTeX commands.
\usepackage{xmpmulti}
\usepackage{algorithm,algpseudocode}

\makeatother

\usepackage{babel}
\addto\shorthandsspanish{\spanishdeactivate{~<>}}

\addto\captionsenglish{\renewcommand{\definitionname}{Definition}}
\addto\captionsspanish{\renewcommand{\algorithmname}{Algoritmo}}
\addto\captionsspanish{\renewcommand{\definitionname}{Definici�n}}
\providecommand{\definitionname}{Definition}

\begin{document}
\title{\selectlanguage{spanish}%
Problema <<>>}
\author{\selectlanguage{spanish}%
Autor1, Autor2, Autor3}
\date{\selectlanguage{spanish}%
\today}
\maketitle
\begin{abstract}
En este documento se presenta el an�lisis del algoritmo ...., .
\end{abstract}

\part{An�lisis y dise�o del problema}

\section{An�lisis}

El problema, informalmente, se puede describir como: ...

\section{Dise�o}

Con las observaciones presentadas en el an�lisis anterior, podemos
escribir el dise�o de un algoritmo que solucione el problema . A veces
este dise�o se conoce como el <<contrato>> del algoritmmos o las
<<precondiciones>> y <<poscondiciones>> del algoritmo. El dise�o
se compone de entradas y salidas:
\begin{defn*}
Entradas:
\end{defn*}
\selectlanguage{english}%
\begin{enumerate}
\item Definici�n entrada 1
\item Definici�n entrada 2
\end{enumerate}
\selectlanguage{spanish}%
~~~~
\begin{defn*}
Salidas:
\end{defn*}
\begin{enumerate}
\item Definici�n salida 1
\item Definici�n salida 2
\end{enumerate}
~~~~~

\part{Algoritmos}

\section{Opci�n algoritmo 1}

\subsection{Algoritmo}

Este algoritmo se basa en ...

\begin{algorithm}
\begin{algorithmic}[1]

\Procedure{ALG}{$S$}

  \State$n\leftarrow|S|$

\For{$a\leftarrow1$ $\mathbf{to}$ $n$}

\State$b\leftarrow x^{a}$

\EndFor

  \State\Return{$b$}

\EndProcedure

\end{algorithmic}

\caption{\foreignlanguage{english}{Nombre}}
\end{algorithm}


\subsection{Complejidad}

El algoritmo \noun{X} tiene �rden de complejidad . �Por qu�?

\subsection{Invariante}
\begin{itemize}
\item \textbf{\uline{Inicio}}: 
\item \textbf{\uline{Avance}}: 
\item \textbf{\uline{Terminaci�n}}
\end{itemize}

\subsection{Notas de implementaci�n}

\part{Comparaci�n de los algoritmos}\selectlanguage{english}%

\end{document}
